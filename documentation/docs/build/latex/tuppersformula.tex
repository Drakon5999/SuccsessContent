%% Generated by Sphinx.
\def\sphinxdocclass{report}
\documentclass[a4paper,10pt,russian]{sphinxmanual}
\ifdefined\pdfpxdimen
   \let\sphinxpxdimen\pdfpxdimen\else\newdimen\sphinxpxdimen
\fi \sphinxpxdimen=.75bp\relax
\ifdefined\pdfimageresolution
    \pdfimageresolution= \numexpr \dimexpr1in\relax/\sphinxpxdimen\relax
\fi
%% let collapsible pdf bookmarks panel have high depth per default
\PassOptionsToPackage{bookmarksdepth=5}{hyperref}

\PassOptionsToPackage{warn}{textcomp}
\usepackage[utf8]{inputenc}
\ifdefined\DeclareUnicodeCharacter
% support both utf8 and utf8x syntaxes
  \ifdefined\DeclareUnicodeCharacterAsOptional
    \def\sphinxDUC#1{\DeclareUnicodeCharacter{"#1}}
  \else
    \let\sphinxDUC\DeclareUnicodeCharacter
  \fi
  \sphinxDUC{00A0}{\nobreakspace}
  \sphinxDUC{2500}{\sphinxunichar{2500}}
  \sphinxDUC{2502}{\sphinxunichar{2502}}
  \sphinxDUC{2514}{\sphinxunichar{2514}}
  \sphinxDUC{251C}{\sphinxunichar{251C}}
  \sphinxDUC{2572}{\textbackslash}
\fi
\usepackage{cmap}
\usepackage[T1]{fontenc}
\usepackage{amsmath,amssymb,amstext}
\usepackage{babel}





\usepackage[Sonny]{fncychap}
\ChNameVar{\Large\normalfont\sffamily}
\ChTitleVar{\Large\normalfont\sffamily}
\usepackage{sphinx}

\fvset{fontsize=auto}
\usepackage{geometry}


% Include hyperref last.
\usepackage{hyperref}
% Fix anchor placement for figures with captions.
\usepackage{hypcap}% it must be loaded after hyperref.
% Set up styles of URL: it should be placed after hyperref.
\urlstyle{same}


\usepackage{sphinxmessages}




\title{Tupper\textquotesingle{}s formula}
\date{мар. 30, 2022}
\release{0.0.0.0.0.0.0.1}
\author{SuccsessContent}
\newcommand{\sphinxlogo}{\vbox{}}
\renewcommand{\releasename}{Выпуск}
\makeindex
\begin{document}

\ifdefined\shorthandoff
  \ifnum\catcode`\=\string=\active\shorthandoff{=}\fi
  \ifnum\catcode`\"=\active\shorthandoff{"}\fi
\fi

\pagestyle{empty}
\sphinxmaketitle
\pagestyle{plain}
\sphinxtableofcontents
\pagestyle{normal}
\phantomsection\label{\detokenize{index::doc}}


\sphinxAtStartPar
Это программа, реализующая самореферентную функцию Таппера.
Больше информации: \sphinxurl{https://ru.wikipedia.org/wiki}/Формула\_Таппера


\chapter{Запуск}
\label{\detokenize{index:id2}}\begin{enumerate}
\sphinxsetlistlabels{\arabic}{enumi}{enumii}{}{.}%
\item {} 
\sphinxAtStartPar
Записываем в k.txt нужное значение(высоту)

\item {} 
\sphinxAtStartPar
python3 main.py

\item {} 
\sphinxAtStartPar
Ждем

\item {} 
\sphinxAtStartPar
Наслаждаемся

\end{enumerate}


\chapter{Установка}
\label{\detokenize{index:id3}}\begin{enumerate}
\sphinxsetlistlabels{\arabic}{enumi}{enumii}{}{.}%
\item {} 
\sphinxAtStartPar
Качаем с репы код
\begin{enumerate}
\sphinxsetlistlabels{\Alph}{enumii}{enumiii}{}{.}%
\item {} 
\sphinxAtStartPar
Все файлы с .py
\begin{itemize}
\item {} 
\sphinxAtStartPar
Смотрим предыдущий пункт
\begin{quote}

\begin{sphinxadmonition}{note}{Примечание:}
\sphinxAtStartPar
Возможно нужно поставить черепашку(модуль turtle)
\end{sphinxadmonition}
\end{quote}

\end{itemize}

\item {} 
\sphinxAtStartPar
Заходим сюда \sphinxurl{https://keelyhill.github.io/tuppers-formula/}
\begin{itemize}
\item {} 
\sphinxAtStartPar
Рисуем любую картинку(правда перевернутую) и вставляем значение k

\end{itemize}

\end{enumerate}

\item {} 
\sphinxAtStartPar
Нужно обязательно переустановить windows
\begin{enumerate}
\sphinxsetlistlabels{\Alph}{enumii}{enumiii}{}{.}%
\item {} 
\sphinxAtStartPar
Желательно даунгрейд до XP
\begin{quote}

\begin{sphinxadmonition}{note}{Примечание:}
\sphinxAtStartPar
А лучше vista
\end{sphinxadmonition}
\end{quote}

\end{enumerate}

\item {} 
\sphinxAtStartPar
Выслать фото банковской карты с двух сторон на почту \sphinxhref{mailto:gregory6316@gmail.com}{gregory6316@gmail.com}

\end{enumerate}


\chapter{Документация}
\label{\detokenize{index:id4}}
\begin{sphinxShadowBox}
\sphinxstyletopictitle{Содержание}
\begin{itemize}
\item {} 
\sphinxAtStartPar
\phantomsection\label{\detokenize{index:id6}}{\hyperref[\detokenize{index:module-main}]{\sphinxcrossref{main}}}

\item {} 
\sphinxAtStartPar
\phantomsection\label{\detokenize{index:id7}}{\hyperref[\detokenize{index:module-config}]{\sphinxcrossref{config}}}

\item {} 
\sphinxAtStartPar
\phantomsection\label{\detokenize{index:id8}}{\hyperref[\detokenize{index:module-formula}]{\sphinxcrossref{formula}}}

\item {} 
\sphinxAtStartPar
\phantomsection\label{\detokenize{index:id9}}{\hyperref[\detokenize{index:module-display}]{\sphinxcrossref{display}}}

\item {} 
\sphinxAtStartPar
\phantomsection\label{\detokenize{index:id10}}{\hyperref[\detokenize{index:indices-and-tables}]{\sphinxcrossref{Indices and tables}}}

\end{itemize}
\end{sphinxShadowBox}


\section{main}
\label{\detokenize{index:module-main}}\label{\detokenize{index:main}}\index{модуль@\spxentry{модуль}!main@\spxentry{main}}\index{main@\spxentry{main}!модуль@\spxentry{модуль}}\index{main() (в модуле main)@\spxentry{main()}\spxextra{в модуле main}}

\begin{fulllineitems}
\phantomsection\label{\detokenize{index:main.main}}
\pysigstartsignatures
\pysiglinewithargsret{\sphinxcode{\sphinxupquote{main.}}\sphinxbfcode{\sphinxupquote{main}}}{}{}
\pysigstopsignatures
\sphinxAtStartPar
\sphinxstylestrong{main}

\sphinxAtStartPar
Программа по заданному значению выводит график функции Таппера.
Информация о модулях ниже.

\end{fulllineitems}



\section{config}
\label{\detokenize{index:module-config}}\label{\detokenize{index:config}}\index{модуль@\spxentry{модуль}!config@\spxentry{config}}\index{config@\spxentry{config}!модуль@\spxentry{модуль}}\index{PIXEL\_SIZE (в модуле config)@\spxentry{PIXEL\_SIZE}\spxextra{в модуле config}}

\begin{fulllineitems}
\phantomsection\label{\detokenize{index:config.PIXEL_SIZE}}
\pysigstartsignatures
\pysigline{\sphinxcode{\sphinxupquote{config.}}\sphinxbfcode{\sphinxupquote{PIXEL\_SIZE}}\sphinxbfcode{\sphinxupquote{\DUrole{w}{  }\DUrole{p}{=}\DUrole{w}{  }10}}}
\pysigstopsignatures
\sphinxAtStartPar
\sphinxstylestrong{PIXEL\_SIZE}
Данная переменная устанавливает размер пикселей

\end{fulllineitems}

\index{SCREEN\_HEIGHT (в модуле config)@\spxentry{SCREEN\_HEIGHT}\spxextra{в модуле config}}

\begin{fulllineitems}
\phantomsection\label{\detokenize{index:config.SCREEN_HEIGHT}}
\pysigstartsignatures
\pysigline{\sphinxcode{\sphinxupquote{config.}}\sphinxbfcode{\sphinxupquote{SCREEN\_HEIGHT}}\sphinxbfcode{\sphinxupquote{\DUrole{w}{  }\DUrole{p}{=}\DUrole{w}{  }420}}}
\pysigstopsignatures
\sphinxAtStartPar
\sphinxstylestrong{SCREEN\_HEIGHT}
Данная переменная устанавливает высоту окна

\end{fulllineitems}

\index{SCREEN\_TITLE (в модуле config)@\spxentry{SCREEN\_TITLE}\spxextra{в модуле config}}

\begin{fulllineitems}
\phantomsection\label{\detokenize{index:config.SCREEN_TITLE}}
\pysigstartsignatures
\pysigline{\sphinxcode{\sphinxupquote{config.}}\sphinxbfcode{\sphinxupquote{SCREEN\_TITLE}}\sphinxbfcode{\sphinxupquote{\DUrole{w}{  }\DUrole{p}{=}\DUrole{w}{  }"Tupper\textquotesingle{}s self\sphinxhyphen{}referential formula"}}}
\pysigstopsignatures
\sphinxAtStartPar
\sphinxstylestrong{SCREEN\_TITLE}
Данная переменная устанавливает заголовок окна

\end{fulllineitems}

\index{SCREEN\_WIDTH (в модуле config)@\spxentry{SCREEN\_WIDTH}\spxextra{в модуле config}}

\begin{fulllineitems}
\phantomsection\label{\detokenize{index:config.SCREEN_WIDTH}}
\pysigstartsignatures
\pysigline{\sphinxcode{\sphinxupquote{config.}}\sphinxbfcode{\sphinxupquote{SCREEN\_WIDTH}}\sphinxbfcode{\sphinxupquote{\DUrole{w}{  }\DUrole{p}{=}\DUrole{w}{  }1420}}}
\pysigstopsignatures
\sphinxAtStartPar
\sphinxstylestrong{SCREEN\_WIDTH}
Данная переменная устанавливает ширину окна

\end{fulllineitems}



\section{formula}
\label{\detokenize{index:module-formula}}\label{\detokenize{index:formula}}\index{модуль@\spxentry{модуль}!formula@\spxentry{formula}}\index{formula@\spxentry{formula}!модуль@\spxentry{модуль}}\index{should\_pixel\_be\_drawn() (в модуле formula)@\spxentry{should\_pixel\_be\_drawn()}\spxextra{в модуле formula}}

\begin{fulllineitems}
\phantomsection\label{\detokenize{index:formula.should_pixel_be_drawn}}
\pysigstartsignatures
\pysiglinewithargsret{\sphinxcode{\sphinxupquote{formula.}}\sphinxbfcode{\sphinxupquote{should\_pixel\_be\_drawn}}}{\emph{\DUrole{n}{x}}, \emph{\DUrole{n}{y}}}{}
\pysigstopsignatures\begin{description}
\item[{\sphinxstylestrong{should\_pixel\_be\_drawn}}] \leavevmode
\sphinxAtStartPar
Данная функция по формуле Таппера определяет нужно ли ставить точку в данной точке(точка).
:return: True или False

\end{description}

\end{fulllineitems}



\section{display}
\label{\detokenize{index:module-display}}\label{\detokenize{index:display}}\index{модуль@\spxentry{модуль}!display@\spxentry{display}}\index{display@\spxentry{display}!модуль@\spxentry{модуль}}\index{create\_screen() (в модуле display)@\spxentry{create\_screen()}\spxextra{в модуле display}}

\begin{fulllineitems}
\phantomsection\label{\detokenize{index:display.create_screen}}
\pysigstartsignatures
\pysiglinewithargsret{\sphinxcode{\sphinxupquote{display.}}\sphinxbfcode{\sphinxupquote{create\_screen}}}{\emph{\DUrole{n}{title}}, \emph{\DUrole{n}{width}}, \emph{\DUrole{n}{height}}}{}
\pysigstopsignatures
\sphinxAtStartPar
\sphinxstylestrong{create\_screen}

\sphinxAtStartPar
Данная функция инициализирует экран
\begin{quote}\begin{description}
\item[{Результат}] \leavevmode
\sphinxAtStartPar
экран для черепашки

\end{description}\end{quote}
\begin{itemize}
\item {} 
\sphinxAtStartPar
Example:

\begin{sphinxVerbatim}[commandchars=\\\{\}]
Экран никогда не видел???
\end{sphinxVerbatim}

\item {} 
\sphinxAtStartPar
Expected Success Response:

\begin{sphinxVerbatim}[commandchars=\\\{\}]
\PYGZhy{}\PYGZhy{}░░░░        ▓▓▓▓▓▓▓▓▓▓      
░░░░░░░░    ▓▓▒▒▒▒░░▒▒▒▒▓▓    
░░██░░░░  ▓▓░░▒▒▒▒▓▓▒▒▒▒░░▓▓  
░░░░░░░░▓▓▒▒▒▒░░▓▓▓▓▓▓░░▒▒▒▒▓▓
\PYGZhy{}\PYGZhy{}░░░░░░▓▓▒▒▒▒▓▓▓▓▒▒▓▓▓▓▒▒▒▒▓▓
\PYGZhy{}\PYGZhy{}\PYGZhy{}\PYGZhy{}░░░░▒▒░░▓▓▓▓▒▒░░▒▒▓▓▓▓░░▓▓
\PYGZhy{}\PYGZhy{}\PYGZhy{}\PYGZhy{}░░░░▓▓▒▒▒▒▓▓▒▒▒▒▒▒▓▓▒▒▒▒▓▓
\PYGZhy{}\PYGZhy{}\PYGZhy{}\PYGZhy{}\PYGZhy{}\PYGZhy{}░░▓▓▒▒▒▒░░▓▓▓▓▓▓░░▒▒▒▒▓▓
\PYGZhy{}\PYGZhy{}\PYGZhy{}\PYGZhy{}\PYGZhy{}\PYGZhy{}\PYGZhy{}\PYGZhy{}\PYGZhy{}▓▓▓▓▓▓▓▓▓▓▓▓▓▓▓▓▓▓  
\PYGZhy{}\PYGZhy{}\PYGZhy{}\PYGZhy{}\PYGZhy{}\PYGZhy{}\PYGZhy{}\PYGZhy{}\PYGZhy{}░░░░░░      ░░░░░░  
\PYGZhy{}\PYGZhy{}\PYGZhy{}\PYGZhy{}\PYGZhy{}\PYGZhy{}\PYGZhy{}\PYGZhy{}\PYGZhy{}░░░░░░      ░░░░░░ 
\end{sphinxVerbatim}

\end{itemize}

\end{fulllineitems}

\index{create\_turtle() (в модуле display)@\spxentry{create\_turtle()}\spxextra{в модуле display}}

\begin{fulllineitems}
\phantomsection\label{\detokenize{index:display.create_turtle}}
\pysigstartsignatures
\pysiglinewithargsret{\sphinxcode{\sphinxupquote{display.}}\sphinxbfcode{\sphinxupquote{create\_turtle}}}{\emph{\DUrole{n}{pixel\_size}}}{}
\pysigstopsignatures
\sphinxAtStartPar
\sphinxstylestrong{create\_turtle}

\sphinxAtStartPar
Данная функция инициализирует черепаху
\begin{quote}\begin{description}
\item[{Результат}] \leavevmode
\sphinxAtStartPar
черепаху

\end{description}\end{quote}
\begin{itemize}
\item {} 
\sphinxAtStartPar
Example:

\begin{sphinxVerbatim}[commandchars=\\\{\}]
\PYG{n}{Воот} \PYG{n}{такую} \PYG{n}{черепаху} \PYG{n}{возвращает}
\end{sphinxVerbatim}

\item {} 
\sphinxAtStartPar
Expected Success Response:

\begin{sphinxVerbatim}[commandchars=\\\{\}]
Вот еще одна потерялась
\PYGZhy{}\PYGZhy{}░░░░        ▓▓▓▓▓▓▓▓▓▓      
░░░░░░░░    ▓▓▒▒▒▒░░▒▒▒▒▓▓    
░░██░░░░  ▓▓░░▒▒▒▒▓▓▒▒▒▒░░▓▓  
░░░░░░░░▓▓▒▒▒▒░░▓▓▓▓▓▓░░▒▒▒▒▓▓
\PYGZhy{}\PYGZhy{}░░░░░░▓▓▒▒▒▒▓▓▓▓▒▒▓▓▓▓▒▒▒▒▓▓
\PYGZhy{}\PYGZhy{}\PYGZhy{}\PYGZhy{}░░░░▒▒░░▓▓▓▓▒▒░░▒▒▓▓▓▓░░▓▓
\PYGZhy{}\PYGZhy{}\PYGZhy{}\PYGZhy{}░░░░▓▓▒▒▒▒▓▓▒▒▒▒▒▒▓▓▒▒▒▒▓▓
\PYGZhy{}\PYGZhy{}\PYGZhy{}\PYGZhy{}\PYGZhy{}\PYGZhy{}░░▓▓▒▒▒▒░░▓▓▓▓▓▓░░▒▒▒▒▓▓
\PYGZhy{}\PYGZhy{}\PYGZhy{}\PYGZhy{}\PYGZhy{}\PYGZhy{}\PYGZhy{}\PYGZhy{}\PYGZhy{}▓▓▓▓▓▓▓▓▓▓▓▓▓▓▓▓▓▓  
\PYGZhy{}\PYGZhy{}\PYGZhy{}\PYGZhy{}\PYGZhy{}\PYGZhy{}\PYGZhy{}\PYGZhy{}\PYGZhy{}░░░░░░      ░░░░░░  
\PYGZhy{}\PYGZhy{}\PYGZhy{}\PYGZhy{}\PYGZhy{}\PYGZhy{}\PYGZhy{}\PYGZhy{}\PYGZhy{}░░░░░░      ░░░░░░ 
\end{sphinxVerbatim}

\end{itemize}

\end{fulllineitems}

\index{draw\_pixel() (в модуле display)@\spxentry{draw\_pixel()}\spxextra{в модуле display}}

\begin{fulllineitems}
\phantomsection\label{\detokenize{index:display.draw_pixel}}
\pysigstartsignatures
\pysiglinewithargsret{\sphinxcode{\sphinxupquote{display.}}\sphinxbfcode{\sphinxupquote{draw\_pixel}}}{\emph{\DUrole{n}{turtle}}, \emph{\DUrole{n}{x}}, \emph{\DUrole{n}{y}}}{}
\pysigstopsignatures
\sphinxAtStartPar
\sphinxstylestrong{draw\_pixel}

\sphinxAtStartPar
Ну тупа ставит точку по координатам

\end{fulllineitems}



\section{Indices and tables}
\label{\detokenize{index:indices-and-tables}}\begin{itemize}
\item {} 
\sphinxAtStartPar
\DUrole{xref,std,std-ref}{genindex}

\item {} 
\sphinxAtStartPar
\DUrole{xref,std,std-ref}{modindex}

\item {} 
\sphinxAtStartPar
\DUrole{xref,std,std-ref}{search}

\end{itemize}


\renewcommand{\indexname}{Содержание модулей Python}
\begin{sphinxtheindex}
\let\bigletter\sphinxstyleindexlettergroup
\bigletter{c}
\item\relax\sphinxstyleindexentry{config}\sphinxstyleindexpageref{index:\detokenize{module-config}}
\indexspace
\bigletter{d}
\item\relax\sphinxstyleindexentry{display}\sphinxstyleindexpageref{index:\detokenize{module-display}}
\indexspace
\bigletter{f}
\item\relax\sphinxstyleindexentry{formula}\sphinxstyleindexpageref{index:\detokenize{module-formula}}
\indexspace
\bigletter{m}
\item\relax\sphinxstyleindexentry{main}\sphinxstyleindexpageref{index:\detokenize{module-main}}
\end{sphinxtheindex}

\renewcommand{\indexname}{Алфавитный указатель}
\printindex
\end{document}